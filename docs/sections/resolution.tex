\section{Clock Resolution}

To begin with, we need tools to measure the time.

The Linux operating system provides multiple system calls to measure the time, in different resolutions.

As per the POSIX manual\cite{posix_clock_gettime}, the function \texttt{gettimeofday} is obsolete so we chose \texttt{clock\_gettime} instead.

Also, the x86 CPU provides \texttt{rdtscp} instruction to give the CPU timestamp, in terms of TSC frequency.

The resolution is the smallest possible increase of the clock. In order to measure the resolution, we tried to create minimal possible differences between two time measurements. We create such difference by inserting an line of assembly \texttt{inc r12} into the code. And the result shows we are producing

However, along with \texttt{clock\_gettime}, function \texttt{clock\_getres} is provided for user to query the resolution of time. As we will demonstrated, it produces the same result as ours.