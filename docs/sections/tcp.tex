\section{TCP}
TCP uses sockets to communicate the between the hosts which is used across local or remote communication. TCP uses byte-ordered reliable communication mechanism which is widely supported across today's operating systems. In order to measure the latency and throughput of the TCP stream sockets, packet sizes of 4, 16, 64, 256, 1K, 4K, 16K, 64K, 256K, and 512K were considered.

We hypothesized that the latency will proportionally increase with the packet size, and the throughput will also increase as the packet sizes are increased. To get the most accurate representation of the results the test for each individual transfer unit was repeated 1 million times and the minimum number amongst those 1 million trials was considered to be the value closest to the real value.

One of the issues encountered while measuring TCP latency was that when using the read and write function calls of the Rust language on TCP stream, the functions were reading and writing random bytes of data which were different than the expected values. This caused discrepancies in the read and write sequence of the server and the client model. To remedy this problem, we switched to different function calls, read_all and write_all, which helped synchronize the flow of data between the client and the server.

--results


